\documentclass[10pt,a5paper]{article}
\usepackage{graphicx}
\usepackage{hyperref}
\usepackage[utf8x]{inputenc}

\begin{document}

\begin{titlepage}
\begin{center}

\includegraphics[scale=0.25]{sakuralogo.png}
\vfill

\huge
4MB PCMCIA SRAM

User's Manual

\vspace*{1cm}

\normalsize
Low cost Fast RAM expansion for Amiga 600/1200.

\vspace*{5cm}

\today

\end{center}
\end{titlepage}

\section*{Overview}

Thank you for purchasing Sakura 4MB PCMCIA SRAM expansion! This product has the following features:

\begin{itemize}
	\item Adds 4MB of Fast RAM to Amiga 600 or Amiga 1200.
	\item Accelerates unexpanded Amiga 1200 to 1.67 MIPS (according to SysInfo)
	\item Built using modern, high performance 55ns SRAM ICs 
	\item Very simple installation - just insert the card into PCMCIA slot on the left side of Amiga
	\item Compatible with all PCMCIA friendly accelerators and memory expansions\footnote{We did our best to ensure that this card works on as many configurations as possible, but some accelerator and Fast RAM expansion designs are inherently incompatible with PCMCIA slot if more than 4MB of memory is installed. Note that PCMCIA SRAM should work with them anyway, if less than 4MB Fast was installed on the turbo card. Of course, if you don’t have any accelerator, you don’t have to worry about that.}
	\item Open source design.
	\item Made by Amigans for Amigans! 
\end{itemize}

To reduce the price, card is delivered without any kind of case. The case might be available in the future from 3rd party vendor.

\section*{Installation}

The installation process is very easy. To install the expansion perform the following steps:

\begin{itemize}
	\item Power down your Amiga.
	\item Slide the expansion card into PCMCIA slot, located on the left side of your computer. The card's components should be facing upwards.
	\item Turn on your Amiga and enjoy additional 4MB of memory.
\end{itemize}

Your Amiga should start as normal, albeit a bit faster. After installation you can confirm that the board is working correctly, by checking amount of available memory with the {\tt avail} command. Popular SysInfo tool will report the additional 4MB of memory under {\tt card.resource} name.

\section*{Technical details}

Revision 2.x Sakura boards are built around Alliance Memory AS6C3216, which is 55ns 32 megabit SRAM chip with 16-bit data bus.

Control signals for SRAM are generated by Xilinx XC9536XL CPLD, based on PCMCIA slot access signals.

This expansion is a completely autoconfiguring device. Memory is automatically added to the system memory pool by Kickstart. Autoconfiguration process is done according to the Card Information Structure (CIS) provided by the card, which is parsed by {\tt card.resource}. Naturally, this process only takes place during system startup, so inserting card into Amiga when it is running does not cause memory to be added.

Sakura card CIS contains optimal settings that allow best possible PCMCIA slot peformance:
\begin{itemize}
	\item On Amiga 600 Gayle chip is configured for 570ns access time by default, which results in performance similar to Fast RAM expansions attached on top of 68000 CPU.
	\item On Amiga 1200 the default configuration is 250ns access time, but Kickstart automatically reconfigures Gayle for 100ns access, according to CIS. This results in some performance increase compared to an unexpanded Amiga 1200. Contrary to popular opinions, Amiga 1200 equipped with high performance, correclty configured PCMCIA SRAM expansion (like this one) is performing better than Chip RAM only system.
\end{itemize}

CIS ROM is also implemented within XC9536XL CPLD.

Sakura expansion is electrically compatible with PCMCIA 2.1 standard, however physical dimensions of the card are smaller. While in theory this card should work with other devices, using it with computers other than Amiga was never tested and is discouraged.

\section*{Acknowledgements}

This expansion was designed by Radosław ,,strim'' Kujawa and Jarosław ,,jarob'' Bieliński. 

The original idea was suggested on PPA.pl forum by RomanWorkshop. 

Thanks to everyone who preordered the board - you made this project happen!

The card conforms to RoHS standard. 

All new cards sold through our exclusive dealer, RetroAmi, are covered by 24 months warranty. In case of necessary service repairs please contact the shop directly. Do not attempt to repair the card yourself, it will void warranty. Please save the invoice/bill as a proof of transaction.

\end{document}

